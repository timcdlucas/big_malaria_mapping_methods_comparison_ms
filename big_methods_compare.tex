
%% BioMed_Central_Tex_Template_v1.06
%%                                      %
%  bmc_article.tex            ver: 1.06 %
%                                       %

%%IMPORTANT: do not delete the first line of this template
%%It must be present to enable the BMC Submission system to
%%recognise this template!!

%%%%%%%%%%%%%%%%%%%%%%%%%%%%%%%%%%%%%%%%%
%%                                     %%
%%  LaTeX template for BioMed Central  %%
%%     journal article submissions     %%
%%                                     %%
%%          <8 June 2012>              %%
%%                                     %%
%%                                     %%
%%%%%%%%%%%%%%%%%%%%%%%%%%%%%%%%%%%%%%%%%


%%%%%%%%%%%%%%%%%%%%%%%%%%%%%%%%%%%%%%%%%%%%%%%%%%%%%%%%%%%%%%%%%%%%%
%%                                                                 %%
%% For instructions on how to fill out this Tex template           %%
%% document please refer to Readme.html and the instructions for   %%
%% authors page on the biomed central website                      %%
%% http://www.biomedcentral.com/info/authors/                      %%
%%                                                                 %%
%% Please do not use \input{...} to include other tex files.       %%
%% Submit your LaTeX manuscript as one .tex document.              %%
%%                                                                 %%
%% All additional figures and files should be attached             %%
%% separately and not embedded in the \TeX\ document itself.       %%
%%                                                                 %%
%% BioMed Central currently use the MikTex distribution of         %%
%% TeX for Windows) of TeX and LaTeX.  This is available from      %%
%% http://www.miktex.org                                           %%
%%                                                                 %%
%%%%%%%%%%%%%%%%%%%%%%%%%%%%%%%%%%%%%%%%%%%%%%%%%%%%%%%%%%%%%%%%%%%%%

%%% additional documentclass options:
%  [doublespacing]
%  [linenumbers]   - put the line numbers on margins

%%% loading packages, author definitions

%\documentclass[twocolumn]{bmcart}% uncomment this for twocolumn layout and comment line below
\documentclass{bmcart}

%%% Load packages
%\usepackage{amsthm,amsmath}
%\RequirePackage{natbib}
%\RequirePackage[authoryear]{natbib}% uncomment this for author-year bibliography
%\RequirePackage{hyperref}
\usepackage[utf8]{inputenc} %unicode support
%\usepackage[applemac]{inputenc} %applemac support if unicode package fails
%\usepackage[latin1]{inputenc} %UNIX support if unicode package fails
\usepackage{amsmath}
\usepackage{graphicx}
\usepackage{subfig}
\usepackage{amsmath}




%%%%%%%%%%%%%%%%%%%%%%%%%%%%%%%%%%%%%%%%%%%%%%%%%
%%                                             %%
%%  If you wish to display your graphics for   %%
%%  your own use using includegraphic or       %%
%%  includegraphics, then comment out the      %%
%%  following two lines of code.               %%
%%  NB: These line *must* be included when     %%
%%  submitting to BMC.                         %%
%%  All figure files must be submitted as      %%
%%  separate graphics through the BMC          %%
%%  submission process, not included in the    %%
%%  submitted article.                         %%
%%                                             %%
%%%%%%%%%%%%%%%%%%%%%%%%%%%%%%%%%%%%%%%%%%%%%%%%%


%\def\includegraphic{}
%\def\includegraphics{}



%%% Put your definitions there:
\startlocaldefs
\endlocaldefs


%%% Begin ...
\begin{document}

%%% Start of article front matter
\begin{frontmatter}

\begin{fmbox}
\dochead{Research}

%%%%%%%%%%%%%%%%%%%%%%%%%%%%%%%%%%%%%%%%%%%%%%
%%                                          %%
%% Enter the title of your article here     %%
%%                                          %%
%%%%%%%%%%%%%%%%%%%%%%%%%%%%%%%%%%%%%%%%%%%%%%

\title{Thirty methods for mapping malaria: geostatics, machine learning, transfer learning and new covariates}

%%%%%%%%%%%%%%%%%%%%%%%%%%%%%%%%%%%%%%%%%%%%%%
%%                                          %%
%% Enter the authors here                   %%
%%                                          %%
%% Specify information, if available,       %%
%% in the form:                             %%
%%   <key>={<id1>,<id2>}                    %%
%%   <key>=                                 %%
%% Comment or delete the keys which are     %%
%% not used. Repeat \author command as much %%
%% as required.                             %%
%%                                          %%
%%%%%%%%%%%%%%%%%%%%%%%%%%%%%%%%%%%%%%%%%%%%%%

\author[
   addressref={aff1},                   % id's of addresses, e.g. {aff1,aff2}
   corref={aff1},                       % id of corresponding address, if any
  % noteref={n1},                        % id's of article notes, if any
   email={timcdlucas@gmail.com}   % email address
]{\inits{TCDL}\fnm{Tim CD} \snm{Lucas}}
\author[
   addressref={aff1}
]{\inits{RA}\fnm{Rohan} \snm{Arambepola}}
\author[
   addressref={aff1}
]{\inits{RA}\fnm{Tasmin} \snm{Symons}}
\author[
   addressref={aff1}
]{\inits{AKN}\fnm{Anita K} \snm{Nandi}}
\author[
   addressref={aff1}
]{\inits{JR}\fnm{Jennifer} \snm{Rozier}}
\author[
   addressref={aff1}
]{\inits{PA}\fnm{Punam} \snm{Amratia}}
\author[
   addressref={aff1}
]{\inits{EGC}\fnm{Elisabeth G} \snm{Chestnutt}}
\author[
   addressref={aff1}
]{\inits{SHK}\fnm{Suzanne H} \snm{Keddie}}
\author[
   addressref={aff1}
]{\inits{ELC}\fnm{Emma L} \snm{Collins}}
\author[
   addressref={aff1}
]{\inits{SR}\fnm{Susan} \snm{Rumisha}}
\author[
   addressref={aff1}
]{\inits{AP}\fnm{Andre} \snm{Python}}
\author[
   addressref={aff1}
]{\inits{AB}\fnm{Amelia} \snm{Bertozzi-Villa}}
\author[
   addressref={aff1}
]{\inits{PH}\fnm{Penelope} \snm{Hancock}}
\author[
   addressref={aff1}
]{\inits{EC}\fnm{Ewan} \snm{Cameron}}
\author[
   addressref={aff1}
]{\inits{PWG}\fnm{Pete W} \snm{Gething}}
\author[
   addressref={aff1}
]{\inits{DJW}\fnm{Daniel J} \snm{Weiss}}



%%%%%%%%%%%%%%%%%%%%%%%%%%%%%%%%%%%%%%%%%%%%%%
%%                                          %%
%% Enter the authors' addresses here        %%
%%                                          %%
%% Repeat \address commands as much as      %%
%% required.                                %%
%%                                          %%
%%%%%%%%%%%%%%%%%%%%%%%%%%%%%%%%%%%%%%%%%%%%%%

\address[id=aff1]{%                           % unique id
  \orgname{Big Data Institute}, % university, etc
  \street{Old Road Campus},                     %
  %\postcode{}                                % post or zip code
  \city{Oxford},                              % city
  \cny{UK}                                    % country
}

%%%%%%%%%%%%%%%%%%%%%%%%%%%%%%%%%%%%%%%%%%%%%%
%%                                          %%
%% Enter short notes here                   %%
%%                                          %%
%% Short notes will be after addresses      %%
%% on first page.                           %%
%%                                          %%
%%%%%%%%%%%%%%%%%%%%%%%%%%%%%%%%%%%%%%%%%%%%%%

\begin{artnotes}
%\note{Sample of title note}     % note to the article
%\note[id=n1]{Equal contributor} % note, connected to author
\end{artnotes}

\end{fmbox}% comment this for two column layout

%%%%%%%%%%%%%%%%%%%%%%%%%%%%%%%%%%%%%%%%%%%%%%
%%                                          %%
%% The Abstract begins here                 %%
%%                                          %%
%% Please refer to the Instructions for     %%
%% authors on http://www.biomedcentral.com  %%
%% and include the section headings         %%
%% accordingly for your article type.       %%
%%                                          %%
%%%%%%%%%%%%%%%%%%%%%%%%%%%%%%%%%%%%%%%%%%%%%%

\begin{abstractbox}

\begin{abstract} % abstract
\parttitle{Background} %if any

The mapping of risk is a core task in malaria epidemiology and the results from modelled maps are used to estimate national totals of prevalence and incidence as well as being used to optimise the delivery of malaria interventions.
Many methods for mapping malaria have been developed over the decades and these methodological advancements have influenced both malaria epidemiology and the wider field of geostatistics.
However, new methods for mapping malaria, and new covariates that can be used in these models are rarely compared in a systematic and thorough way.


\parttitle{Methods} %if any
Here we compare a large number of covariates and thirty methods for the mapping of malaria.
As a study system we consider malaria prevalence surveys in Africa from 2000--2018 at a spatial resolution of 5 $\times$ 5 km.
The covariates considered include a standard suite of remotely sensed variables such as temperature, precipitation, enhanced vegetation index, elevation and travel time to cities.
To this set we add environmental suitability for different Anopheles species, housing quality, additional human movement variables and world bank indicators.
The methods compared include machine learning models, model based geostatistics, model ensembles such as stacked generalisation and random Fourier features.
We also develop some novel methods such as spatially varying model stacking, causal inference feature selection and data fusion methods for combining malaria prevalence surveys with mosquito occurrence points, entomological inoculation rate data and lymphatic filariasis prevalence surveys.

\parttitle{Results} %if any

We find that...

\parttitle{Conclusion} %if any
This suggests that.
Different models have different benefits.

\end{abstract}

%%%%%%%%%%%%%%%%%%%%%%%%%%%%%%%%%%%%%%%%%%%%%%
%%                                          %%
%% The keywords begin here                  %%
%%                                          %%
%% Put each keyword in separate \kwd{}.     %%
%%                                          %%
%%%%%%%%%%%%%%%%%%%%%%%%%%%%%%%%%%%%%%%%%%%%%%

\begin{keyword}
\kwd{Malaria}
\kwd{Mapping}
\kwd{Geospatial modelling}
\kwd{Machine Learning}
\end{keyword}

% MSC classifications codes, if any
%\begin{keyword}[class=AMS]
%\kwd[Primary ]{}
%\kwd{}
%\kwd[; secondary ]{}
%\end{keyword}

\end{abstractbox}
%
%\end{fmbox}% uncomment this for twcolumn layout

\end{frontmatter}

%%%%%%%%%%%%%%%%%%%%%%%%%%%%%%%%%%%%%%%%%%%%%%
%%                                          %%
%% The Main Body begins here                %%
%%                                          %%
%% Please refer to the instructions for     %%
%% authors on:                              %%
%% http://www.biomedcentral.com/info/authors%%
%% and include the section headings         %%
%% accordingly for your article type.       %%
%%                                          %%
%% See the Results and Discussion section   %%
%% for details on how to create sub-sections%%
%%                                          %%
%% use \cite{...} to cite references        %%
%%  \cite{koon} and                         %%
%%  \cite{oreg,khar,zvai,xjon,schn,pond}    %%
%%  \nocite{smith,marg,hunn,advi,koha,mouse}%%
%%                                          %%
%%%%%%%%%%%%%%%%%%%%%%%%%%%%%%%%%%%%%%%%%%%%%%

%%%%%%%%%%%%%%%%%%%%%%%%% start of article main body
% <put your article body there>

%%%%%%%%%%%%%%%%
%% Background %%
%%
\section*{Background}

malaria is important and so is mapping


new methods or covariates rarely compared on single dataset
but see

covariates as important as model

new covariates include...

new methods include...

Methods have different benefits.

here we find...


\section*{Methods}

\subsection*{Overview}

We aimed to test both new covariates and new methods but did not want to test every combination of covariate and method.
Therefore we first used a small set of methods (elastic net, random forest and model based geostatistics) to test a range of covariate sets.
From these first comparisons we then defined two covariate sets to take forward and test with every method.

\subsection*{Model Comparison and Performance Metrics}


We randomly selected 20\% of the data to hold out as a validation set.
Our main performance metric was weighted mean absolute error (MAE) with the data being weighted by the survey sample size.
We tested whether differences in performance were significant using a weighted Mann-Whitny U test with the data again weighted by survey sample size.

It is difficult to compare methods in a fully fair way, especially when some methods have many tuning parameters.
All tuning parameters were selected with an inner validation step.
That is, the 80\% of data remaining after data splitting was then split again into training and testing sets.
However, there is no truly consistent way to decide how many sets of tuning parameters to try for each model.
We therefore aimed to adhere to two principles.
Firstly, we aimed to replicate realistic modelling behaviour.
Notably, modellers generally do not select priors in Bayesian spatial models using cross-validation due to both run time and for philosophical reasons so we also fitted these models by setting priors \emph{a priori}.
Secondly, we aimed to try enough parameter sets that we were fairly confident that we would not find massive performance increases by broadening the search.
For models with relatively few parameters this can be assessed visually.
The aim of the study is not to squeeze out every possible marginal gain as the relative results will change slightly with different datasets.

\subsection*{Prevalence data}

The prevalence survey data were extracted from the Malaria Atlas Project prevalence survey database using only data from 2000 onwards \cite{bhatt2015effect, guerra2007assembling, pfeffer2018ma}.
We standardised days by diagnostic using models from \cite{}.
The prevalence points were then standardised to an age range of 2--10 using the model from \cite{smith2007standardizing}.
The final dataset contains 55,914 surveys in 44,842 distinct locations and represents samples from 5,687,304 individuals.

\subsection*{Models for covariate comparison}


We used three models to compare different covariate sets; elastic net, random forest and model based geostatics.
The models were chosen too be relatively simple but still able to learn from both spatial signal (mbg) and non-linear relationships between covariates and disease (Random Forest).
Furthermore, we wanted methods that could easily handle the increased number of covariates (elastic net and random forest).

We used elastic net \cite{enet} and optimised the penalty and LASSO/ridge fraction using an inner validation holdout set of 34% of the data.
Similarly for Random Forest we optimised the split rule, mtry and the minimum node size using random search and a 34% holdout set.

The model based geostatistics model was given by the hierarchical Bayesian equations
\begin{align}
P_i \sim Binom(prev_i, exa_i)
prev_i = \operatorname{logit}\left(\beta_0 + \beta X + u_{s,t}\left(\rho, \sigma, \alpha\right)\right)
\end{align}

where $P_i$ is the observed number of positive cases, $prev_i$ is the estimated prevalence and $exa_i$ is the number of people tested.
The linear predictor includes an intercept, $\beta_0$, regression coefficients, $\beta$, along with a matrix if covariates, $X$.
We give $\beta_0$ and $\beta$ normal priors with mean 0 and sd of Todo.
The random effect, $u_{s, t}\left(\rho, \sigma, \alpha)\right)$ is a spatiotemporal effect.
It is a spatial random field with penalised complexity priors \cite{fuglstad2018constructing} for the range and nominal sd such that $p(\rho<10) = 0.01$ and $p(\sigma > 0.4) = 0.01$.
The temporal component is modelled by splitting the range of dates into 5 bins with each bin getting its own random effect.
The correlation between the fields in each year bin is modelled as an ar1 process with a penalised complexity prior on the correlation between year bins such that $p(\alpha < 0.9) = 0.01$.
These models were fitted with INLA \cite{inla}.

%spde <- inla.spde2.pcmatern(mesh = %mesh, alpha = 2,
%                            prior.range = c(10, 0.01), %prior.sigma = c(0.4, 0.01))


%h.spec <- list(theta=list(prior='pccor1', %param=c(0, 0.9)))
%form <- as.formula(paste('pf_pos ~ b0 %+ 0 + f(field, model=spde, group = %field.group, control.group = %list(model="ar1", hyper=h.spec)) + ', %fixed))




\subsection*{Further covariates}


%%% baseline


%%% mosquito


%%% accessibility


%%% interventions and housing


%%% world Bank



%%% Weiss


We used the covariates selected by a large covariate selection process described by \cite{weiss2015re}.
In \cite{weiss2015re} they started with some standard covariates similar to the baseline set used here but then considered different spatial and temporal aggregations, transformations and interaction terms before selecting down to 20 covariates.
Instead of adding these covariates to the baseline covariate set, we tested it as a standalone set as the


\subsection*{Further methods}


%%% Geostatistics

We tested standard model based geostatistics models with a spatiotemporal field
 month rw2 and year rw2.
 Monthly term not modelled as effectively as some other methods \cite{nguyen2020mapping}.
These were all fitted with INLA \cite{INLA}.
random Fourier features.


%%% Machine learning methods \label{mlmethods}

We tested a suite of standard machine learning models (M5--10): elastic net \cite{}, projection pursuit regression \cite{}, random forest \cite{ranger},  boosted regression trees \cite{xgboost}, Bayesian additive regressive trees \cite{} and single layer neutral networks \cite{nnet}.
All models were trained with caret \cite{caret} using an inner three fold cross-validation scheme to select tuning parameters.
When possible we used the one standard error rule \cite{breiman1984classification} whereby the model selected to be the final model was the simplest model with performance within one standard error of the best performing model.
For models with fewer than three tuning parameters we used grid search while for models with three or more parameters we used random search.
Parameter values for testing were largely selected using caret's default ranges.

%%% Spatial machine learning

To include spatial information with machine learning methods we followed a modified version of the method by \cite{}.
We choose 1000 locations by k means clustering of the PR data.
We then created 1000 additional covariates such that covariate $j$ for data point $i$ was given by $1 / distance (x_i, x_j)$ where $x_i$ and $x_j$ are the spatial locations of data point $i$ and k cluster centre $j$ respectively.
Using these additional covariates we fitted a random forest model (M11) as above.


%%% Deep learning

Three layer network


%%% Ensemble methods

We tested four methods that ensemble the predictions from section \ref{mlmethods} (M12--16).
The first three are standard methods, model averaging weighting the models by their inner cross-validation MAE \cite{}, using Random forest to stack the models \cite{} and using model based geostatistics to stack the models (i.e. stacked generalisation as ok \cite{bhatt2017improved}.

The final ensemble method is, to our knowledge, new.
In this method we imagine that the performance of machine learning models will differ over space.
We therefore fit a spatially varying intercepts model with the coefficients forced to be above one.
The greater than one constraint forces the model to perform model averaging.
By enforcing a greater than one constraint rather than a true sum to one constraint we also avoid the need for an additional, purely spatial random effect.


%%% Transfer learning and data fusion

Transfer learning and data fusion methods aim to use information from directly it indirectly related data.
Here we examine the use of mosquito occurrence data, entomological inoculation rate data and lymphatic filariasis prevalence surveys (M17--20).



\section*{Results}

\section*{Discussion}

%\bibliographystyle{apalike} %or any other style you like

\bibliographystyle{plain}


\bibliography{Malaria}



\end{document}
