%%%%%%%%%%%%%%%%%%%%%%%%%%%%%%%%%%%%%%%%%
% Professional Formal Letter
% LaTeX Template
% Version 2.0 (12/2/17)
%
% This template originates from:
% http://www.LaTeXTemplates.com
%
% Authors:
% Brian Moses
% Vel (vel@LaTeXTemplates.com)
%
% License:
% CC BY-NC-SA 3.0 (http://creativecommons.org/licenses/by-nc-sa/3.0/)
%
%%%%%%%%%%%%%%%%%%%%%%%%%%%%%%%%%%%%%%%%

%----------------------------------------------------------------------------------------
%	PACKAGES AND OTHER DOCUMENT CONFIGURATIONS
%----------------------------------------------------------------------------------------

\documentclass[10pt, a4paper]{letter} % Set the font size (10pt, 11pt and 12pt) and paper size (letterpaper, a4paper, etc)


\usepackage[nodayofweek]{datetime}


\usepackage[hidelinks]{hyperref}

\setlength\longindentation{5cm}

\input{structure.tex} % Include the file that specifies the document structure

%\longindentation=0pt % Un-commenting this line will push the closing "Sincerely," and date to the left of the page

%----------------------------------------------------------------------------------------
%	YOUR INFORMATION
%----------------------------------------------------------------------------------------

\Who{Dr Tim C.D. Lucas} % Your name

\Title{} % Your title, leave blank for no title

\authordetails{
	Big Data Institute\\ % Your department/institution
	University of Oxford\\
	OX3 7LF,, U.K.\\ % Your city, zip code, country, etc
	timcdlucas@gmail.com\\ % Your email address
	+44 (0) 7415863536\\ % Your phone number
	\\
	\today
}

%----------------------------------------------------------------------------------------
%	HEADER CONTENTS
%----------------------------------------------------------------------------------------

\logo{oxford-logo.png} % Logo filename, your logo should have square dimensions (i.e. roughly the same width and height), if it does not, you will need to adjust spacing within the HEADER STRUCTURE block in structure.tex (read the comments carefully!)

\headerlineone{UNIVERSITY} % Top header line, leave blank if you only want the bottom line

\headerlinetwo{OF OXFORD} % Bottom header line

%----------------------------------------------------------------------------------------

\begin{document}

%----------------------------------------------------------------------------------------
%	TO ADDRESS
%----------------------------------------------------------------------------------------

\begin{letter}{
Prof. Marcel Hommel
Editor-in-Chief
Malaria Journal
}

%----------------------------------------------------------------------------------------
%	LETTER CONTENT
%----------------------------------------------------------------------------------------

\opening{Dear Prof. Marcel Hommel,}

I am pleased to submit the article ``'' to be considered for publication in the Malaria Journal.
There are two files: the main manuscript (tim-lucas-model-compare.pdf), the supplementary code (tim-lucas-model-compare.zip) and the supplementary material (tim-lucas-model-compare-si.pdf).

Mapping of disease risk has been a core component of malaria epidemiology for decades and the methods developed for malaria epidemiology have been influential in the broader field of geostatistics.
The estimation of malaria risk allows us to calculate national level disease metrics such as incidence and prevalence and also allows for interventions to be targeted more optimally.
While methods for malaria mapping are continually being developed, the various models are rarely compared in a rigorous manner.
Furthermore, a fact neglected in epidemiology but well known in the predictive modelling community, is that a good covariate can often improve model performance more than a better model.
Again, the assessment of new covariates for malaria mapping is rarely done  in a general manner.

In this paper we have compared a large number of mapping methods and a large number of covariates to assess the predictive performance of both.
We use as our study system the mapping of malaria prevalence, annually between 2000 and 2018, at 5km resolution, across Africa.
We compare many state of the art modelling methods such as stacked generalisation, random fourier feature approximations to Gaussian processes, model based geostatistics and machine learning methods.
We also develop a number of new methods such as spatially varying model stacking and data fusion methods for combining mosquito, EIR and lymphatic filariasis data.
We also compare a range of newly collected or estimated covariates that hitherto have been rarely used in malaria mapping.
These covariates include housing quality, mosquito suitability, world bank indicators and new metrics of human movement.

We hope that this paper will act as a guide to the variety of mapping methods available, a benchmark against which future methods can be compared as well as the reference for the newly developed methods.


\closing{Yours sincerely,}

%----------------------------------------------------------------------------------------
%	OPTIONAL FOOTNOTE
%----------------------------------------------------------------------------------------

% Uncomment the 4 lines below to print a footnote with custom text
%\def\thefootnote{}
%\def\footnoterule{\hrule}
%\footnotetext{\hspace*{\fill}{\footnotesize\em Footnote text}}
%\def\thefootnote{\arabic{footnote}}

%----------------------------------------------------------------------------------------

\end{letter}

\end{document}
